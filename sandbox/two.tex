\setuppagenumbering[location={footer, middle}]
\setupbodyfont[10pt]
\setuphead
  [title]
  [align=center, style=\ss, size=\tfa, color=red, after={\blank[2*big]}]

\setuphead
  [chapter]
  [style=\ssc, after={\blank[2*big]}, align=right]

\usemodule[filter]

   \defineexternalfilter
     [markdown]
     [filtercommand={pandoc -f \externalfilterparameter{format} -t context
                     -i \externalfilterinputfile\space
                     -o \externalfilteroutputfile},
      format=markdown,
      directory=output]

\starttext



\startstandardmakeup
\startalignment[center] % or \centerline{...}
    {\definedfont[Bold at 40pt] Title}
\stopalignment
\stopstandardmakeup

\setuppagenumbering[conversion=romannumerals,left={--~},right={~--}]
\startstandardmakeup
{\tfd  Table of contents}
\placecontent
\stopstandardmakeup

\page

\chapter{Introduction}
\input knuth

\placefigure {Library OS}
             {\externalfigure [swlayers][width=10cm]}
\startitemize[n, columns, 5][compact]
\item Asia
\item Africa
\item Australia
\item Antarctica
\item Europe
\item North America
\item South America
\item Arctic
\stopitemize



\setuppagenumbering[conversion=arabicnumerals,left={--~},right={~--}]
\startcolumns[n=2,tolerance=verytolerant]
  Hasselt is an old Hanseatic City, situated 12 ̃km north of Zwolle at
  the river Zwartewater.
  The city has a long history since obtaining the city charter around
  1252. Part and parcel of this history can be traced back to a large
  number of monuments to be admired in the city center.

There you will find the St. Stephanus church, a late gothic church
  dating back to 1479 with a magnificent organ. The former Municipal
  Building is situated on The Market Place. Constituted between 1500
  and 1550 it houses a large collection of weapons, amongst which one
  of the largest collection of black powder guns (haakhussen) in the
  whole world should be mentioned.
  Furthermore there is a corn windmill ‘The Swallow’, dating back to
  1748 as well as the ‘Stenendijk’, a unique embankment and the last
  shell limekiln in Europe still in full operation.
  The city center with the townmoat adorned by lime-trees, the Van
  Stolkspark and the hustle and bustle at the docks are ideally suited
  for a stroll.
  The area around Hasselt is also worth mentioning. In wintertime
  polder Mastenbroek harbours large numbers of geese. In summertime the
  hamlets Genne, Streukel and Celle\-mui\-den form, together with the very
  rare lapwing flowers (Lat. Fritillaria meleagris) found on the banks
  of the river Zwatewater, the ideal surroundings for walking or
  cycling trips.
  Hasselt also is a very important center for watersports. The lakes of
  northwest Overijssel, the river IJssel, the Overijsselse Vecht and
  the Randmeren are within easy reach from the yacht harbour ‘De
  Molenwaard’. Sailing, fishing, swimming and canoeing can be fully
  enjoyed in Hasselt.
  Furthermore some events of special interest should be
  mentioned. Every year at the end of August Hasselt celebrates the
  ‘Eui Festival’ (hay festival).
 \stopcolumns

\starttable[|c|c|]
\HL
\NC \bf Year \VL \bf citizens \NC\SR
\HL
\NC 1989     \VL 289    \NC\FR
\NC 1990     \VL 292    \NC\MR
\NC 1981     \VL 289    \NC\LR
\HL
\stoptable

\section{Literature survey}

Some \color[blue]{text} and/or \color[green]{graphics}.

\startmarkdown
  Hello **world**. A big advantage of a lightweight markup language
  like markdown is that it is easy to convert it into other
  markups--html, rtf, epub, etc. For that reason, I key in markdown in
  a separate file rather in a start-stop environment of a \TeX\ file.
\stopmarkdown

\blank[3*big]

\startcolumns[n=2]

\processmarkdownfile{software-engineering.md}

\stopcolumns

\stoptext
